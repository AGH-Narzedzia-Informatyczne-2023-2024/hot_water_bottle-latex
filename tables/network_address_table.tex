\begin{tabular}{|c|c|c|c|}
    \hline
        \multicolumn{4}{|c|}{\textbf{główny router budynku A}} \\
    \hline
        sieć & maska & brama & interfejs \\
    \hline
        \multicolumn{4}{|c|}{(w przypadku awarii połączenia z routerem A:)} \\
        0.0.0.0     &   0.0.0.0         &   96.0.0.0    &   eth1    \\
        98.0.0.0    &   255.0 0 0       &   0.0.0.0     &   eth1    \\
    \hline
        \multicolumn{4}{|c|}{(Gdy wszystko działa:)} \\
        0.0.0.0     &   0.0.0.0         &   193.1.42.1  &   eth0    \\
        193.1.42.0  &   255.255.255.0   &   0.0.0.0     &   eth0    \\
    \hline
        100.0.0.0   &   255.0.0.0       &   0.0.0.0     &   eth1    \\
        96.0.0.0    &   255.0.0.0       &   0.0.0.0     &   eth2    \\
        98.32.0.0   &   255.240.0.0     &   0.0.0.0     &   eth3    \\
        98.64.0.0   &   255.240.0.0     &   0.0.0.0     &   eth4    \\
        98.128.0.0  &   255.240.0.0     &   0.0.0.0     &   eth5    \\
        98.192.0.0  &   255.240.0.0     &   0.0.0.0     &   eth6    \\
    
    \hline
    
%\label{tab:netA}
\end{tabular}