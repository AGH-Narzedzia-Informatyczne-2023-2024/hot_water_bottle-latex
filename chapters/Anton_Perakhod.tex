
\section{Anton Perakhod}
 

\label{sec:Anton Perakhod}

Image of the modern Google logo (see Figure~\ref{fig:google_logo} on page \pageref {fig:google_logo}).

\begin{figure}[htbp]
 \centering{ \includegraphics[width=0.8\textwidth]{pictures/google_logo}}
    \caption{Image of the modern Google logo}
    \label{fig:google_logo}
\end{figure}
\begin{table}[htbp]
\centering
\begin{tabular}{|c| |c| |c| |c|} \hline
 Col1 & Col2 & Col3 & Col4 \\ [0.5ex] \hline\hline
 11 & 11 & 22 & 33 \\ \hline
 2 & 8 & 78 & 12 \\ \hline
 3 & 545 & 345 & 3 \\ \hline
 7 & 56 & 972 & 1\\ \hline
 5 & 88 & 124 & 2 \\ \hline
\end{tabular}
\label{tab:table_A}
\caption{Przykładowa tabela}
\end{table}
\begin{itemize}
\item 
Here is also a worth seeing equation:

\[\cos (2\alpha) = \cos^2 \alpha - \sin^2 \alpha\]

\item
Another one is here:
 \end{itemize}

  \[x = a_0 + \cfrac{1}{a_1 
          + \cfrac{1}{a_2 
          + \cfrac{1}{a_3 + \cfrac{1}{a_4} } } } \]
\newpage

\begin{flushleft}
  \setlength{\parindent}{1em} 
 \title{\Large \textbf{History of Google}}
 \vspace{0.2cm}
  \par
 \textbf{Google} (Figure~\ref{fig:google_logo} on page \pageref{fig:google_logo} was founded on September 4, 1998, by American computer scientists Larry  Page and Sergey Brin while they were PhD students at Stanford University in \underline{California}. Together they own about 14 of its publicly listed shares and control 56 of its stockholder voting power through super-voting stock. \par
 \vspace{0.1cm}
  They called this search engine Backrub. Soon after, Backrub was renamed \textbf{Google} (phew).\emph{The name was a play on the mathematical expression for the number 1 followed by 100 zeros and aptly reflected Larry and Sergey's mission “to organize the world's information and make it universally accessible and useful}.
\end{flushleft}
\textbf{\large Subjects for which there will be exams in the first semester:}
\begin{enumerate}
    \item Analiza matematyczna
    \item Algebra liniowa i geometria analityczna
    \item Architektury komputerów
\end{enumerate}
\vspace{0.7cm}
\textbf{\Large Subjects for which there will be exams in the first semester:}
    \begin{itemize}
    \item Analiza matematyczna
    \item Algebra liniowa i geometria analityczna
    \item Architektury komputerów
\end{itemize}
